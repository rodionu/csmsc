\documentclass[11pt]{article}	%Default was article
\usepackage{fullpage,graphicx,listings}		%Reduces line margins to a more reasonable 1". also graphix

\title{ECE-477 - Hardware Applications in C\\Digital Multimeter} %EDIT THESE! the \\ gives a new line!
\author{Cameron S. McGary, Joel M. Castro} 
\date{\today} 			% The \today command fills in the date when printed, you can type what ever you want here!


\begin{document} 		% We being to put stuff on a page! weee!
\maketitle 			% Output all the title information here


\thispagestyle{empty} 		% We dont want any page numbers on THIS page
\clearpage			% We aren't in report document class, so we need to clear the page buffer. 

\tableofcontents
\thispagestyle{empty} 		% We dont want any page numbers on THIS page
\setcounter{page}{0} 		% Reset the page count to zero. 


\newpage			% This starts a new page, handy to keep sections at the top of pages. LaTeX is smart about the rest of the text, so trust it!

\section{Introduction}

The objective of this project is to create a working controller that allows a ball on a track to roll from one end of the track to the other without falling off. The control system created here utilizes a four term compensator to drive the plant and a unity feedback scheme. The input to the plant is shaped with a low pass filter to buffer the response of the system. All controller zeroes are placed at least a radial distance of 1 from all plant poles to avoid any unintended pole/zero cancellation.
		% The additional sections work like this
\clearpage
\section{Setting up port I/O}



The AVR ATMEGA88P (and many other series of the AVR chip) include integrated pull-up resistors coupled to the I/O pins for use in active-low inputs. This is very useful as it allows us to make use of these pins without adding complication to the circuit with external resistors, and without having to be concerned over whether or not the state of these pins is known.

The PORTC register contains the A/D converters on most pins, this register will be used for A/D only. The specific input will vary with the function of the entire device itself (DC voltage, DC current work on separate pins, etc.)

We determine the A/D converter to activate by reading the inputs on PORTB. Only 4 of the PORTB pins are used (PB1, PB2, PB3, PB4) as only 4 bits of information are needed. Initializing these pins is done by specifing the data direction register to input on those 4 pins, and writing logic ones to them. Writing a logic one to an input pin causes the AVR to activate the internal pull-up resistor to that pin.

\begin{lstlisting}		%This is for inserting code bits straight into Latex, syntax highlighting

DDRB = 0b11110001;
PORTB = 0b00001110;
\end{lstlisting}

\clearpage
\section{Simulations}

SIMULATIONS

\subsection{Sim1}

Simulation one

\subsection{Sim2}

Simulation two
\clearpage
\section{Controller Prefilter}

The step response of the controller exhibited a major problem that when the step was applied, the ball would roll off the wrong side of the track. This can be seen in figure \ref{3:fig:stepresponse} where the purple line (representing the ball position) starts at the initial condition and goes towards the target, but jumps backwards briefly. This is the controller compensating for the fact that the ball is initially rolling faster than the controller would prefer (due to track initial positioning) and jumping to correct it. With no prefilter, this compensation becomes extreme leading to a large enough overshoot to drop the ball off the edge of the track. The prefilter prevents this from happening by dampening the step input to the circuit. Undoubtedly the response would be faster without this input signal shaping, but it is necessary for proper performance.

Here the signal shaper will be examined, its impact on the step source as well as on a pulse source.

\begin{figure}[h!t]
	\centering
		\includegraphics[width=0.70\textwidth]{pics/lowpassfilter}
	\caption{Prefilter response, Input step 0 to -2.5 at t=1 sec)}
	\label{3:fig:lowpassfilter}
\end{figure}

The filter damps the step response so it can be sent to the controller. This is that the controller is actually responding to, the user would not see this signal. The step response provided earlier is purely what the user sees in terms of input and output.

Switching the signal to a pulse (to examine response time to non-step input signals gives a far different response. The settling time would likely be almost fast enough to position itself reasonably from this waveform, but the filter attenuates the up and down impulses of the step waveform heavily, slowing the response. Initial conditions were unchanged for this pulse response test.

\begin{figure}[h!t]
	\centering
		\includegraphics[width=0.70\textwidth]{pics/pulseresponse}
	\caption{Controller response to pulse input}
	\label{3:fig:pulseresponse}
\end{figure}
\FloatBarrier

Now the controller looks almost as if it is trying to follow a sinusoidal wave, but if we take a look at what the prefilter is doing to the signal, we see that it is not too far off from the reference signal it is receiving.

\begin{figure}[h!t]
	\centering
		\includegraphics[width=0.70\textwidth]{pics/lowpasspulse}
	\caption{Filtered pulse input}
	\label{3:fig:lowpasspulse}
\end{figure}
\FloatBarrier

This is actually fairly close to what the controller is trying to produce. This prefilter does negatively impact the settling speed of the response, but it damps the controller's sudden inputs in the wrong direction which was considered more important given the sensitivity of the item being controlled.


\clearpage
\section{Conclusion}

Most of the components and concepts presented in this report were tested and working. The typecasting and string print functions on the LCD did in fact work, and the math should have been done correctly. However, this math is subject to a calibration constant (which would have been set at compile time after testing the completed circuit with real input voltages) which was not set up at the time. The LCD demonstrated communication and output through an 8 bit, 8 character interface, using a function interface that allowed the main program to (largely) ignore how the function worked.

We still do not know what the issue was with the ADCs on the chip, as we were unable to get them to work even with old code. We plan to strip down the board and rebuild it and test again, as we concluded that it was not an issue with our specific chip. There are a few ideas we still had that unfortunately due to time constraints (and once again, ADC issues) that were not implementable.
Obviously not all these items may be workable, and it requires that the original base project works first, but they're just ideas for where this could go:

\begin{itemize}
\item Move LCD to PORT B, allowing for concurrent operation with serial USART.
\item Adjust LCD writing function to utilize all 16 characters on display.
\item Add more LCD writing functions to allow use of functions similar to sprintf() without manually casting.
\item Implement an actual switch, or a three position switch for mode using two pins (DIP switch?)
\item Switch to high-voltage operational amplifiers to ensure safety of parts
\item Perform AC analysis using FFT, instead of the simple averaging method.
\item Some sort of concurrent operation mode to detect power usage using both meters
\end{itemize}


\end{document}			% Must be included to end the document. 